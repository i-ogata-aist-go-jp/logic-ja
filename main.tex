\documentclass[dvipdfmx,autodetect-engine]{jsarticle}
\bibliographystyle{plain} 

\title{ lattice を定義する2つの方法: axiomatic identities で定義された lattice と、 join/meet を持つ partially ordered set として定義された ordered lattice が同等であること}
\author{i.ogata@aist.go.jp }

\bibliographystyle{plain} 
\input{packages}
\theoremstyle{definition}
\newtheorem{theorem}{Theorem}[section]
\newtheorem{lemma}[theorem]{Lemma}
\newtheorem{proposition}[theorem]{Proposition}
\newtheorem{fact}[theorem]{fact}
\newtheorem{corollary}[theorem]{Corollary}
\newtheorem{definition}[theorem]{Definition}
\newtheorem{example}[theorem]{Example}
\newtheorem{remark}[theorem]{Remark}

%% newcommands
\newcommand{\defeq}{\mathrel{\mathop:}=}
\newcommand{\eqdef}{\mathrel{\mathop=}:}
\newcommand{\ONE}{\mbox{\boldmath{1}}}
\newcommand{\ZERO}{\mbox{\boldmath{0}}}

% multisets
\makeatletter
\@ifpackageloaded{stix}{%
}{%
  \DeclareFontEncoding{LS2}{}{\noaccents@}
  \DeclareFontSubstitution{LS2}{stix}{m}{n}
  \DeclareSymbolFont{stix@largesymbols}{LS2}{stixex}{m}{n}
  \SetSymbolFont{stix@largesymbols}{bold}{LS2}{stixex}{b}{n}
  \DeclareMathDelimiter{\lBrace}{\mathopen} {stix@largesymbols}{"E8}%
                                            {stix@largesymbols}{"0E}
  \DeclareMathDelimiter{\rBrace}{\mathclose}{stix@largesymbols}{"E9}%
                                            {stix@largesymbols}{"0F}
}
\makeatother

\newcommand{\multiset}[1]{\lBrace {#1} \rBrace}

% Residuated Lattices  RL

\newcommand{\RLone}{\ONE}
\newcommand{\RLzero}{\ZERO}
\newcommand{\RLotimes}{\otimes}
\newcommand{\RLimp}{\multimap}
\newcommand{\RLoplus}{\oplus}
\newcommand{\RLcdot}{\cdotbin}

\newcommand{\RLstandard}{(L,\LTCjoin,\LTCbot,\LTCmeet,\LTCtop,\RLotimes,\RLone,\RLimp)}

\newcommand{\RLotimesab}{a \RLotimes b}
\newcommand{\RLotimesba}{b \RLotimes a}
\newcommand{\RLotimesac}{a \RLotimes c}
\newcommand{\RLotimesca}{c \RLotimes a}
\newcommand{\RLotimescb}{c \RLotimes b}
\newcommand{\RLotimesbc}{b \RLotimes c}
\newcommand{\RLotimesbd}{b \RLotimes d}

\newcommand{\RLimpac}{a \multimap c}
\newcommand{\RLimpab}{a \multimap b}
\newcommand{\RLimpbc}{b \multimap c}
\newcommand{\RLimpad}{a \multimap d}
\newcommand{\RLimpca}{c \multimap a}
\newcommand{\RLimpcb}{c \multimap b}

\newcommand{\RLojcacb}{(\RLotimesca) \LTCjoin (\RLotimescb)}
\newcommand{\RLimcacb}{(\RLimpca) \LTCmeet (\RLimpcb)}

\newcommand{\RLneg}[1]{{{#1}^{\bot}}}
\newcommand{\RLnegP}[1]{{{(#1)}^{\bot}}}
\newcommand{\dRLneg}[1]{{{#1}^{\bot\bot}}}
\newcommand{\dRLnegP}[1]{{(#1)^{\bot\bot}}}
\newcommand{\RLnegA}{\RLneg{a}}
\newcommand{\dRLnegA}{\dRLneg{a}}
\newcommand{\RLnegB}{\RLneg{b}}
\newcommand{\dRLnegB}{\dRLneg{b}}

\newcommand{\RLbigwedge}{\bigvee ( c \RLotimes A) }
\newcommand{\RLbigimp}{\bigwedge ( c \RLimp A)}

% DeMorgan

\newcommand{\DeMorganMeet}{\RLnegP{\RLnegA \LTCmeet \RLnegB}}
\newcommand{\DeMorganJoin}{\RLnegP{\RLnegA \LTCjoin \RLnegB}}
\newcommand{\DeMorganJoinD}{\dRLnegP{\RLnegA \LTCjoin \RLnegB}}

% Lattices LTC
\newcommand{\LTCjoin}{\vee}
\newcommand{\LTCbot}{\bot}
\newcommand{\LTCmeet}{\wedge}
\newcommand{\LTCtop}{\top}
\newcommand{\LTCimp}{\to}
\newcommand{\LTCeq}{\leftrightarrow}
\newcommand{\LTCneg}{\neg}
\newcommand{\LTCdneg}{\neg\neg}
\newcommand{\LTCotimes}{\otimes}
\newcommand{\LTCtrue}{\top}
\newcommand{\LTCfalse}{\bot}

\newcommand{\LTClattice}{(L,\LTCjoin,\LTCbot,\LTCmeet,\LTCtop)}
\newcommand{\LTCheyting}{(L,\LTCjoin,\LTCbot,\LTCmeet,\LTCtop,\LTCimp)}
%
\newcommand{\LTCaxiom}[1]{\AxiomC{$ {#1} \fCenter {#1}$}}
\newcommand{\LTCjoinab}{a \LTCjoin b}
\newcommand{\LTCjoinba}{b \LTCjoin a}
\newcommand{\LTCjoincd}{c \LTCjoin d}
\newcommand{\LTCjoinac}{a \LTCjoin c}
\newcommand{\LTCjoinbc}{b \LTCjoin c}
\newcommand{\LTCjoinbd}{b \LTCjoin d}
\newcommand{\LTCmeetab}{a \LTCmeet b}
\newcommand{\LTCmeetba}{b \LTCmeet a}
\newcommand{\LTCmeetac}{a \LTCmeet c}
\newcommand{\LTCmeetbc}{b \LTCmeet c}
\newcommand{\LTCmeetbd}{b \LTCmeet d}
\newcommand{\LTCmeetca}{c \LTCmeet a}
\newcommand{\LTCmeetcb}{c \LTCmeet b}
\newcommand{\LTCjoincacb}{(\LTCmeetca)\LTCjoin(\LTCmeetcb)}
\newcommand{\LTCmeetaB}{a \LTCmeet B}
%
\newcommand{\LTCimpaa}{a \LTCimp a}
\newcommand{\LTCimpab}{a \LTCimp b}
\newcommand{\LTCimpac}{a \LTCimp c}
\newcommand{\LTCimpca}{c \LTCimp a}
\newcommand{\LTCimpcb}{c \LTCimp b}
\newcommand{\LTCimpcacb}{(\LTCimpca) \LTCmeet (\LTCimpcb)}
\newcommand{\LTCimpbc}{b \LTCimp c}
\newcommand{\LTCimpad}{a \LTCimp d}
\newcommand{\LTCimpaB}{a \LTCimp B}
%

\newcommand{\LTCnega}{\LTCneg{a}}
\newcommand{\LTCnegb}{\LTCneg{b}}

\newcommand{\LTCdnega}{\LTCneg\LTCneg a}
\newcommand{\LTCdnegb}{\LTCneg\LTCneg b}

\newcommand{\LTCvdash}{\Rightarrow}

\newcommand{\LTCnegP}[1]{\LTCneg ({#1})}
\newcommand{\dLTCnegP}[1]{\LTCneg\LTCneg ({#1})}

%
% Lattices and Closures
\newcommand{\cdotbin}{\mathbin{\cdot}}
\newcommand{\closureLTCjoin}{\mathbin{\closure{\LTCjoin}}}
\newcommand{\closureRLotimes}{\mathbin{\closure{\RLotimes}}}
%
\newcommand{\closure}[1]{\overline{#1}}
\newcommand{\closureOP}{\closure{\phantom{i}}}
\newcommand{\dclosure}[1]{\closure{\closure{#1}}}
\newcommand{\closureX}{\closure{x}}
\newcommand{\closureY}{\closure{y}}
\newcommand{\dclosureX}{\dclosure{x}}
\newcommand{\dclosureY}{\dclosure{y}}

\newcommand{\closureA}{\closure{a}}
\newcommand{\closureB}{\closure{b}}
\newcommand{\closureC}{\closure{c}}
\newcommand{\dclosureA}{\dclosure{a}}
\newcommand{\dclosureB}{\dclosure{b}}
%
\newcommand{\principalFilter}[1]{\mathop{\uparrow}(#1)}
\newcommand{\PfilterX}{\principalFilter{x}}
\newcommand{\PfilterY}{\principalFilter{y}}
\newcommand{\CcapPfilterX}{C\cap\!\principalFilter{x}}
\newcommand{\CcapPfilterY}{C\cap\!\principalFilter{y}}
\newcommand{\CcapPfilterCX}{C\cap\!\principalFilter{\closureX}}
%
\newcommand{\kernel}[1]{\oc{#1}}
\newcommand{\kernelOP}{\kernel{\phantom{i}}}
\newcommand{\dkernel}[1]{\kernel{\kernel{#1}}}
\newcommand{\kernelX}{\kernel{x}}
\newcommand{\kernelY}{\kernel{y}}
\newcommand{\dkernelX}{\dkernel{x}}
\newcommand{\dkernelY}{\dkernel{y}}
% \newcommand{\LLunit}{\mbox{\boldmath{1}}}
% \newcommand{\LLzero}{\bot}
% \newcommand{\LLtrue}{\top}
% \newcommand{\LLfalse}{\mbox{\boldmath{0}}}
%%
%% Quantail QT
%
\newcommand{\QTargument}{{(\mbox{\textendash})}}
\newcommand{\QTargumentT}{{(\mbox{\textendash},\mbox{\textendash})}}
\newcommand{\QTemptyset}{\varnothing}

\newcommand{\QTotimes}{\otimes}
\newcommand{\QToplus}{\oplus}
\newcommand{\QTmultimap}{\multimap}
\newcommand{\QTlub}{\bigvee_ {i \in I}}
\newcommand{\QTglb}{\bigwedge_ {i \in I}}
\newcommand{\QTleftAdjoint}{ a { \QTotimes } {\QTargument}}
\newcommand{\QTrightAdjoint}{ a { \QTmultimap } {\QTargument}}

\newcommand{\QTinterpret}[1]{{#1}^{Q}}
\newcommand{\QTinterpretP}[1]{(#1)^{Q}}

% new command for base Quantales
\newcommand{\QTidentity}{\varepsilon}
\newcommand{\QTbottom}{\varphi}

% \newcommand{\QTidentity}{\QTunit}
\newcommand{\QTunit}{\QTinterpret{\ONE}}
\newcommand{\QTzero}{\QTinterpret{\bot}}
\newcommand{\QTtrue}{\QTinterpret{\top}}
\newcommand{\QTfalse}{\QTinterpret{\ZERO}}
%
% Phase Space
%
\newcommand{\PSargument}{{(\mbox{\textendash})}}
\newcommand{\PScdot}{\cdot}

\newcommand{\PSunit}{\interpret{\ONE}}
\newcommand{\PSzero}{\interpret{\bot}}
\newcommand{\PStrue}{\interpret{\top}}
\newcommand{\PSfalse}{\interpret{\ZERO}}

\newcommand{\interpret}[1]{{#1}^{M}}
\newcommand{\interpretP}[1]{(#1)^{M}}

\newcommand{\PSpowerset}[1]{\mathcal{P}(#1)}
\newcommand{\PSidentity}{\varepsilon}
\newcommand{\PSemptyset}{\varnothing}
\newcommand{\PSzz}{{\mathbb{Z}}}
\newcommand{\PSnn}{{\mathbb{N}}}
\newcommand{\PSsetofintegers}{\PSzz}
\newcommand{\PSsetofpint}{\PSzz^{+}}
\newcommand{\PSsetofnn}{\PSnn}
\newcommand{\PSsetofnnint}{\PSzz^{*}}
\newcommand{\PSsetofnint}{\PSzz^{-}}

\newcommand{\PSexpTrivial}{\set{\PSidentity}}
\newcommand{\PSexpStandard}{{\mbox{\boldmath{I}}}^{M}}
\newcommand{\PSexpFull}{{\mbox{\boldmath{J}}}^{M}}
\newcommand{\PSexpStructure}{{\mbox{\boldmath{K}}}^{M}}

\newcommand{\PSlub}{\bigoplus_ {i \in I} X_i}
\newcommand{\PSglb}{\bigwith_ {i \in I} X_i}
\newcommand{\PSsrc}{\bigoplus_ {i \in I}  (Y \otimes X_i)}
\newcommand{\PSpre}{\bigoplus_ {i \in I}  (Y \cdot X_i)}
\newcommand{\PSdist}{Y \otimes(\PSlub)}

\newcommand{\PSintA}{\interpret{\alpha}}
\newcommand{\PSintB}{\interpret{\beta}}
\newcommand{\PSintC}{\interpret{\gamma}}


\newcommand{\PSEtrue}{\mbox{True}}
\newcommand{\PSEfalse}{\mbox{False}}

\newcommand{\valuation}[1]{\llbracket {#1} \rrbracket}
\newcommand{\valuationC}[1]{\llbracket {#1} \rrbracket^{\mathcal{C}}}
\newcommand{\valuationP}[1]{\llbracket {#1} \rrbracket^{\mathcal{P}}}
\newcommand{\valuationE}[1]{\llbracket {#1} \rrbracket^{\mathcal{E}}}
\newcommand{\valuationG}[1]{\llbracket {#1} \rrbracket^{\mathcal{G}}}
% \newcommand{\valuationX}[1]{\llbracket {#1} \rrbracket^{\mathcal{C}}}

%% Attack Tree
\newcommand{\ATrootbox}[1]{\doublebox{#1}}
\newcommand{\ATnonleafbox}[1]{\fbox{#1}}
\newcommand{\ATleafbox}[1]{\ovalbox{#1}}
%%


\newcommand{\LKsystem}{{\sf\bf LK}}
\newcommand{\MALLsystem}{{\sf\bf MALL}}
\newcommand{\HMALLsystem}{{\sf H-{\bf MALL}}}

%
% LinearLogic LL
%
\newcommand{\entails}{\Rightarrow}
\newcommand{\LLwith}{\with}
\newcommand{\LLoplus}{\oplus}
\newcommand{\LLotimes}{\otimes}
\newcommand{\LLimp}{\multimap}

\newcommand{\LLatimesb}{A \otimes B}
\newcommand{\LLatimesc}{A \otimes C}
\newcommand{\LLbplusc}{B \oplus C}
\newcommand{\LLdistForm}{( \LLatimesb) \oplus (\LLatimesc)}
\newcommand{\LLsrcForm}{A \otimes (\LLbplusc)}
\newcommand{\lnegA}{\lneg{A}}
\newcommand{\dlnegA}{\dlneg{A}}
\newcommand{\lnegB}{\lneg{B}}
\newcommand{\dlnegB}{\dlneg{B}}
\newcommand{\LLtop}{\top}
\newcommand{\LLbot}{\bot}
\newcommand{\LLone}{\ONE}
\newcommand{\LLzero}{\ZERO}
%
\newcommand{\LLvdash}{\mbox{ \Large$\Rightarrow$ }}
\newcommand{\LLargument}{{(\mbox{\textendash})}}
\newcommand{\lnegArg}{\lneg{\LLargument}}
\newcommand{\dlnegArg}{\dlneg{\LLargument}}
%
\newcommand{\lneg}[1]{{{#1}^{\bot}}}
\newcommand{\lnegP}[1]{{(#1)^{\bot}}}
\newcommand{\dlneg}[1]{{{#1}^{\bot\bot}}}
\newcommand{\dlnegP}[1]{{(#1)^{\bot\bot}}}
\newcommand{\lnega}{\lneg{a}}
\newcommand{\lnegb}{\lneg{b}}
\newcommand{\dlnega}{\dlneg{a}}
\newcommand{\dlnegb}{\dlneg{b}}
%
\newcommand{\LLleftAdjoint}{ a { \LLotimes } {\LLargument}}
\newcommand{\LLrightAdjoint}{ a { \LLimp } {\LLargument}}
%
\newcommand{\LLatimesbS}{a \otimes b}
\newcommand{\LLatimescS}{a \otimes c}
\newcommand{\LLbpluscS}{b \oplus c}
\newcommand{\LLdistFormS}{( \LLatimesbS) \LLoplus (\LLatimescS)}
\newcommand{\LLsrcFormS}{a \otimes (\LLbpluscS)}
%
\newcommand{\LLaxiom}[1]{\AxiomC{$$}\UnaryInfC{$#1 \fCenter #1$}}

%%
\newcommand{\LLotimesNP}[2]{{#1}\LLotimes{#2}}
\newcommand{\LLotimesP}[2]{(\LLotimesNP{#1}{#2})}
\newcommand{\LLotimesTrinary}[3]{\LLotimesNP{#1}{\LLotimesP {#2}{#3}}}
\newcommand{\LLimpNP}[2]{{#1}\LLimp{#2}}
\newcommand{\LLimpP}[2]{(\LLimpNP{#1}{#2})}
\newcommand{\LLimpTrinary}[3]{\LLimpNP{#1} {\LLimpP {#2}{#3}}}
\newcommand{\LLimpTrinaryP}[3]{(\LLimpTrinary{#1}{#2}{#3})}
\newcommand{\LLwithNP}[2]{{#1}\LLwith{#2}}
\newcommand{\LLwithP}[2]{(\LLwithNP{#1}{#2})}
\newcommand{\LLwithTrinary}[3]{\LLwithNP{#1}{\withP {#2}{#3}}}
\newcommand{\LLoplusNP}[2]{{#1}\LLoplus{#2}}
\newcommand{\LLoplusP}[2]{(\LLoplusNP{#1}{#2})}
\newcommand{\LLoplusTrinary}[3]{\LLoplusNP{#1}{\LLoplusP {#2}{#3}}}

%% Gentzen'ssequent calculus

\newcommand{\LLaxioms}{
\begin{center}
\begin{prooftree}
\RightLabel{Ax}
\AxiomC{} 
\UnaryInfC {$A \fCenter A$}
	\DisplayProof \hskip 72pt
\RightLabel{DNE}
\AxiomC{}
\UnaryInfC {$\dlneg{A} \fCenter A$}
\end{prooftree}
\end{center}
}

\newcommand{\LLcutRule}{
\begin{center}
\begin{prooftree}
\RightLabel{Cut}
\AxiomC {$\Gamma_0 \fCenter A$}
\AxiomC {$\Gamma_1,A \fCenter D $}
\BinaryInfC  {$\Gamma_0,\Gamma_1 \fCenter D$}
\end{prooftree}
\end{center}
}

\newcommand{\LLaddUnitRule}{
\begin{center}
\begin{prooftree}
\RightLabel{L$\LLzero$}
\AxiomC{}
\UnaryInfC{${\Gamma,\LLzero} \fCenter {D}$}
	\DisplayProof \hskip 72pt
\RightLabel{R$\LLtop$}
\AxiomC{}
\UnaryInfC{${\Gamma} \fCenter {\LLtop}$}
\end{prooftree}
\end{center}
}

\newcommand{\LLwithRule}{
\begin{center}
\begin{prooftree}
\RightLabel{L$\with$$(i = 1,2)$}
\AxiomC {$\Gamma,A_i \fCenter D$}
\UnaryInfC  {$\Gamma,A_1 \with A_2 \fCenter D$}
	\DisplayProof \hskip 32pt
\RightLabel{R$\with$}
\AxiomC {$\Gamma \fCenter A$}
\AxiomC {$\Gamma \fCenter B$}
\BinaryInfC  {$\Gamma \fCenter A \with B$}
\end{prooftree}
\end{center}
}

\newcommand{\LLoplusRule}{
\begin{center}
\begin{prooftree}
\RightLabel{L$\oplus$}
\AxiomC {$\Gamma,A \fCenter D$}
\AxiomC {$\Gamma,B \fCenter D$}
\BinaryInfC  {$\Gamma,A\oplus B \fCenter D$}
	\DisplayProof \hskip 32pt
\RightLabel{R$\oplus$$(i = 1,2)$}
\AxiomC {$\Gamma \fCenter A_i$}
\UnaryInfC  {$\Gamma \fCenter A_1 \oplus A_2$}
%
\end{prooftree}
\end{center}
}

\newcommand{\LLmultUnitRule}{
\begin{center}
\begin{prooftree}
\RightLabel{L$\LLone$}
\AxiomC{$\Gamma\fCenter D$}
\UnaryInfC{$\Gamma,\LLone\fCenter D$}
	\DisplayProof \hskip 72pt
\RightLabel{R$\LLone$}
\AxiomC{}
\UnaryInfC{$ \fCenter \LLone$}
\end{prooftree}
\end{center}
}

\newcommand{\LLotimesRule}{
\begin{center}
\begin{prooftree}
\RightLabel{L$\LLotimes$}
\AxiomC {$\Gamma,A,B \fCenter D$}
\UnaryInfC  {$\Gamma,A \LLotimes B \fCenter D$}
	\DisplayProof \hskip 64pt
\RightLabel{R$\LLotimes$}
\AxiomC {$\Gamma_0 \fCenter A$}
\AxiomC {$\Gamma_1 \fCenter B$}
\BinaryInfC  {$\Gamma_0,\Gamma_1 \fCenter A\LLotimes B$}
\end{prooftree}
\end{center}
}

\newcommand{\LLimpRule} {
\begin{center}
\begin{prooftree}
\RightLabel{L$\LLimp$}
\AxiomC {$\Gamma_0 \fCenter A$}
\AxiomC {$\Gamma_1,B \fCenter D$}
\BinaryInfC  {$\Gamma_0,\Gamma_1,A\LLimp B \fCenter D$}
	\DisplayProof \hskip 32pt
\RightLabel{R$\LLimp$}
\AxiomC {$\Gamma,A \fCenter B$}
\UnaryInfC  {$\Gamma \fCenter A\LLimp B$}
\end{prooftree}
\end{center}
}

\newcommand{\LLnegRule}{
\begin{center}
\begin{prooftree}
\RightLabel{L$\lneg{\QTargument}$}
\AxiomC {$\Gamma \fCenter A$}
\UnaryInfC  {$\Gamma,\lneg{A} \fCenter \LLbot$}
	\DisplayProof \hskip 96pt
\RightLabel{R$\lneg{\QTargument}$}
\AxiomC {$\Gamma,A \fCenter \LLbot$}
\UnaryInfC  {$\Gamma \fCenter \lneg{A}$}
\end{prooftree}
\end{center}
}

%%
\newcommand{\LLdRule}{
\AxiomC{$\Gamma,A \fCenter D$}
\RightLabel{D}
\UnaryInfC{$\Gamma, \oc A \fCenter D$}}

\newcommand{\LLpRule}{
\AxiomC{$\oc \Gamma,\fCenter D$}
\RightLabel{P}
\UnaryInfC{$\oc \Gamma, \oc C \fCenter \oc D$}}

\newcommand{\LLwRule}{
\AxiomC{$\Gamma \fCenter D$}
\RightLabel{W}
\UnaryInfC{$\Gamma, \oc C \fCenter D$}}

\newcommand{\LLcRule}{
\AxiomC{$\Gamma,\oc C,\oc C \fCenter D$}
\RightLabel{C}
\UnaryInfC{$\Gamma, \oc C \fCenter D$}}


%% Classical Logic 

\newcommand{\CLwRule}{
\AxiomC{$\Gamma \fCenter D$}
\RightLabel{Weakening}
\UnaryInfC{$\Gamma, C \fCenter D$}}

\newcommand{\CLcRule}{
\AxiomC{$\Gamma,C,C \fCenter D$}
\RightLabel{Contraction}
\UnaryInfC{$\Gamma, C \fCenter D$}}

\newcommand{\CLstructuralRule}{
\begin{center}
\begin{prooftree}
\CLwRule 
	\DisplayProof \hskip 72pt
\CLcRule
\end{prooftree}
\end{center}
}

\newcommand{\CLaddUnitRule}{
\begin{center}
\begin{prooftree}
\RightLabel{L$\LLzero$}
\AxiomC{}
\UnaryInfC{${\LLzero} \fCenter {D}$}
	\DisplayProof \hskip 72pt
\RightLabel{R$\LLtop$}
\AxiomC{}
\UnaryInfC{$ \fCenter \LLtop $}
\end{prooftree}
\end{center}
}


%% HBT Hilbert
\newcommand{\HBTaxiomK}[2] {\LLimpTrinary{#1}{#2}{#1}}

\newcommand{\HBTaxiomW}[2]
{\LLimpNP
	{\LLimpTrinaryP{#1}{#1}{#2}}
	{\LLimpP{#1}{#2}}}
	
\newcommand{\HBTaxiomS}[3]
{\LLimpTrinary
	{\LLimpTrinaryP{#1}{#2}{#3}}
	{\LLimpP{#1}{#2}}
	{\LLimpP{#1}{#3}}}

\newcommand{\HBTaxiomI}[1]{\LLimpNP{#1}{#1}}

\newcommand{\HBTaxiomA}[3]
{\LLimpNP
	{\LLwithP
		{\LLimpP{#1}{#2}}
		{\LLimpP{#1}{#3}}}
	{\LLimpP
		{#1}
		{\LLwithP{#2}{#3}}}}
        
\newcommand{\HBTaxiomB}[3]
{\LLimpTrinary
	{\LLimpP{#1}{#2}}
	{\LLimpP{#2}{#3}}
	{\LLimpP{#1}{#3}}}
    
\newcommand{\HBTaxiomC}[3]
{\LLimpNP
	{\LLimpTrinaryP{#1}{#2}{#3}}
	{\LLimpTrinaryP{#2}{#1}{#3}}}
    
\newcommand{\HBTaxiomRotimes}[2]
{\LLimpTrinary
	{#1}
	{#2}
	{\LLotimesP{#1}{#2}}}
\newcommand{\HBTaxiomLotimes}[3]
{\LLimpNP
	{\LLimpTrinaryP{#1}{#2}{#3}}
	{\LLimpP
    	{\LLotimesP{#1}{#2}}
        {#3}}}
\newcommand{\HBTaxiomRone}{\LLone}
\newcommand{\HBTaxiomLone}[1]
{\LLimpNP
	{#1}
	{\LLimpP{\LLone}{#1}}}
 
\newcommand{\HBTaxiomLwith}[1]
{
{\LLimpNP
	{\LLwithP{{#1}_1}{{#1}_2}}
	{{#1}_1}}
\hskip 6pt , \hskip 6pt 
{\LLimpNP
	{\LLwithP{{#1}_1}{{#1}_2}}
	{{#1}_2}}}

\newcommand{\HBTaxiomRoplus}[1]
{
{\LLimpNP
	{{#1}_1}
	{\LLoplusP{{#1}_1}{{#1}_2}}}
\hskip 6pt , \hskip 6pt 
{\LLimpNP
	{{#1}_2}
	{\LLoplusP{{#1}_1}{{#1}_2}}}}
    
\newcommand{\HBTaxiomLoplus}[3]
{\LLimpNP
	{\LLwithP
		{\LLimpP{#1}{#3}}
		{\LLimpP{#2}{#3}}}
	{\LLimpP
		{\LLoplusP{#1}{#2}}
		{#3}}}
    
\newcommand{\HBTaxiomTop}[1]{\LLimpNP{#1}{\LLtop}}
\newcommand{\HBTaxiomZero}[1]{\LLimpNP{\LLzero}{#1}}

\newcommand{\HBTaxiomDNE}[1]{\LLimpNP{\dlneg{#1}}{#1}}

%%
% HBT combinators
\newcommand{\HBTcombI}{\mbox{\boldmath{I}}}
\newcommand{\HBTcombB}{\mbox{\boldmath{B}}}
\newcommand{\HBTcombC}{\mbox{\boldmath{C}}}
\newcommand{\HBTcombK}{\mbox{\boldmath{K}}}
\newcommand{\HBTcombW}{\mbox{\boldmath{W}}}
%
\newcommand{\HBTcombBtype}[3]{\HBTcombB_{{#1},{#2},{#3}}}
\newcommand{\HBTcombCtype}[3]{\HBTcombC_{{#1},{#2},{#3}}}
\newcommand{\HBTcombKtype}[2]{\HBTcombK_{{#1},{#2}}}
\newcommand{\HBTcombWtype}[2]{\HBTcombW_{{#1},{#2}}}
%
\newcommand{\HBTtermI}[1]{
	\HBTterm
		{\HBTcombI_{#1}}
		{\LLimpNP{#1}{#1}}}
\newcommand{\HBTtermB}[3]{
	\HBTterm
		{\HBTcombB_{{#1},{#2},{#3}}}
		{\LLimpNP
			{\LLimpP{#1}{#2}}
			{\LLimpP
				{\LLimpP{#2}{#3}}
				{\LLimpP{#1}{#3}}}}}
\newcommand{\HBTtermC}[3]{
	\HBTterm
		{\HBTcombC_{{#1},{#2},{#3}}}
		{\LLimpNP
			{\LLimpTrinaryP{#1}{#2}{#3}}
			{\LLimpTrinaryP{#2}{#1}{#3}}}}
%
% HBT  Curry-Howard term calculus
%
\newcommand{\HBTterm}[2]{ {#1}  :  \hskip 6pt  \vdash \hskip 6pt  {#2} }
%
\newcommand{\HBTpair}[2]{\langle{#1},{#2} \rangle}
\newcommand{\HBTtermMP}[4] {
	\RightLabel{Modus Ponens}  
		\AxiomC {$\HBTterm{#4}{#1}$}   
		\AxiomC {$\HBTterm{#3}{\LLimpNP{#1}{#2}}$}
	\BinaryInfC {$\HBTterm {{#3}{#4}} {#2}$}
}
\newcommand{\HBTtermConjunction}[4] {
	\RightLabel{Conjunction}  
		\AxiomC {$\HBTterm{#3}{#1}$} 
		\AxiomC {$\HBTterm{#4}{#2}$}     
	\BinaryInfC {$\HBTterm {\HBTpair{#3}{#4}} {\withNP{#1}{#2}}$}
}




\begin{document}

\maketitle

\def\fCenter{\leq}
\section{Order Theory}

\subsection{partial order}

\begin{definition}[partial order]
Let $A$ be a set. A {\em partial order} on $A$ is a binary relation $\leq$ which is
\begin{enumerate}
\item {\em reflexive}: for all $a \in A$,  $a \leq a$,
\item {\em transitive}:  if $a \leq b$ and $b \leq c$, then $a \leq c$, and
\item {\em antisymmetric}: if $a \leq b$ and $b \leq a$, then $a = b$.
\end{enumerate}
\end{definition}

ここから order theory の証明を証明図( axiom を leaf とする inference rules  の tree )の形で記述する。 partial order の axiom と inference rules は、上記の partial order の定義から、以下の1つの axiom と、2つの inference rules となる。

The reflexive rule is an axiom, while the transitive and the antisymmetrc rules are inference rules. In this article, these axioms and inference rules are represented as; 
\newcommand{\LTCposetRules}{
\begin{center}
\begin{prooftree}
	\AxiomC{} 
    \RightLabel{ref}
    \UnaryInfC{$a \leq a$}
\DisplayProof  \hskip 2cm
	\AxiomC{$a \leq b$} 
    \AxiomC{$b \leq c$}
    \RightLabel{trans}
    \BinaryInfC{$a \leq c$}
\DisplayProof  \hskip 2cm
	\AxiomC{$a \leq b$} 
    \AxiomC{$b \leq a$}
    \RightLabel{antisymm}
    \BinaryInfC{$a = b$}
\end{prooftree}
\end{center}
}

\LTCposetRules

\subsection{join}

\begin{definition}[join]
Let $(A,\leq)$ be a poset, $S$ a subset of $A$. 
We say an element $a \in A$ is a {\em join} 
(or least upper bound) for $S$,
and write $a = \bigvee S$, if
\begin{enumerate}
\item {\em upper bound}:
$\bigvee S$ is an upper bound for $S$,
i.e. $s \leq \bigvee S \hskip 6pt\text{for all}\hskip 4pt s \in S$, and
\item {\em the least upper bound}: 
if $b$ satisfies $\forall s \in S (s \leq b)$, then $\bigvee S \leq b$. 
\end{enumerate}
\end{definition}

The antisymmetry axiom ensures that the join of $S$, if it exists, is unique. 
If $S$ is a two-element set $\set{s,t}$, we write $s \vee t$ for $\bigvee\set{s,t}$. 
If $S$ is the empty set $\emptyset$, we write $\bot$ for $\bigvee\emptyset$. 
$\bot$ is just the least element of $A$,
because every $c \in A$ is an upper bound for $\emptyset$, i.e. $\bot \leq c$. 

\begin{definition}[bounded join-semilattice-ordered poset]
A bounded join-semilattice-ordered poset is a poset $(L,\leq)$ such that;
\begin{enumerate}
\item  each two-element subset $\set{a,b}$ has a join $a \vee b$, and
\item  the empty set $\emptyset$  has a join, and write $\bot = \bigvee\emptyset$.
\end{enumerate}
\end{definition}

Dually, in any poset we can consider the notion of {\em meet} (greater lower bound)
and meet-semilattice-ordered poset. 
We write $\bigwedge S, a \wedge b, \top = \bigwedge\emptyset$ 
for the analogues of $\bigvee S, a \vee b, \bot$. 
Specifically, $\top$ is the greatest element of $A$,
i.e., $c \leq \LTCtop$ for every $c \in A$. 

つまり join と meet の両方を持つ partially ordered set として定義された ordered lattice は、以下の6つの axioms と、2つの inference rules を持つ。

\newcommand{\LTCjoinRules}{
\begin{center}
\begin{prooftree}
\AxiomC{$$} 
  \RightLabel{$\LTCbot$}
\UnaryInfC{$ \LTCbot \fCenter a $}
	\DisplayProof \hskip16pt
\AxiomC{$$} 
\RightLabel{$R1\LTCjoin$}
\UnaryInfC{$ a \fCenter a \LTCjoin b $}
	\DisplayProof \hskip16pt
\AxiomC{$$} 
\RightLabel{$R2\LTCjoin$}
\UnaryInfC{$ b \fCenter a \vee b $}
	\DisplayProof \hskip 32pt
\AxiomC{$a \fCenter c   $} 
\AxiomC{$b \fCenter c   $} 
\RightLabel{$L\LTCjoin$}
\BinaryInfC{$a \LTCjoin b \fCenter c$}
\end{prooftree}
\end{center} 
}

\newcommand{\LTCmeetRules}{
\begin{center}
\begin{prooftree}
\AxiomC{$$} 
\RightLabel{$\LTCtop$}
\UnaryInfC{$ a \fCenter \LTCtop $}
	\DisplayProof \hskip16pt
\AxiomC{$$} 
\RightLabel{$L1\LTCmeet$}
\UnaryInfC{$ a \LTCmeet b  \fCenter a$}
     \DisplayProof \hskip 16pt
\AxiomC{$$} 
\RightLabel{$L2\LTCmeet$}
\UnaryInfC{$ a \LTCmeet b \fCenter b$}
     \DisplayProof \hskip 32pt
\AxiomC{$c \fCenter  a $} 
\AxiomC{$c \fCenter  b $} 
\RightLabel{$R\LTCmeet$}
\BinaryInfC{$c \fCenter a \LTCmeet b$}     
\end{prooftree}
\end{center}
}

\LTCjoinRules
\LTCmeetRules

\subsection{Commutative Monoids}
% magma
\begin{definition} [magma]
A magma is a basic kind of algebraic structure. It consists of a set $M$ accompanied with a binary operation "$\cdotbin$" which is a map: 
$ {\QTargument} { \cdotbin} {\QTargument} : M \times M \longrightarrow M$.  That is, it sends any two elements $a,b \in M$ to another element $a \cdotbin b$ .
%
In set theoretic notation:
\[ \forall a,b \in M    ,      a \cdot b \in M \]
\end{definition}

In this article, we only handle commutative binary operation; 
i.e., $ a \cdot b = b \cdot a$. 

\begin{definition}[commutative monoid]
A commutative {\it monoid} is a triple 
$(M,\cdotbin,\RLone) $  where $\RLone \in M$. 
The following conditions hold:
%
\begin{enumerate} 
\item {\em neutrality law}:  for all  $a \in M$  we have $\RLone \cdotbin a = a$,
\item {\em associative law}:  for every $a,b,c \in M$,
we have $a \cdotbin ( b \cdotbin c) =  (a \cdotbin b) \cdotbin c$,
\item {\em commutative law}:  for every $a,b \in M$ we have $a \cdotbin b = b \cdotbin a$.
\end{enumerate}
\end{definition} 

\def\fCenter{\leq_\LTCjoin}
\subsection{SemiLattices}
%
A semilattices is a commutative monoid in which every element is idempotent. 
%
\begin{definition}[semilattice]
A {\it semilattice} is a triple $(M,\cdotbin,\RLone) $.
The following conditions hold:
%
\begin{enumerate} 
\item {\em commutative monoid}:  $(M,\cdotbin,\RLone)$ is a commutative monoid,
\item {\em idempotent law}:  for every $a \in M$  we have $a \cdotbin a = a$.
\end{enumerate}
\end{definition}

\begin{proposition}
A bounded join-semilattice-ordered poset is a semilattice.
\end{proposition}
\begin{proof}
Let $(L,\leq_\LTCjoin)$ be a join-semilattice-ordered poset.
\begin{enumerate}
\item The neutrality law holds because $\bot$ is the least element of $L$. 
\item It is obvious that the associative, commutative and idempotent law hold.
\end{enumerate}
\end{proof}
%
Similarly, a bounded meet-semilattice-ordered poset is also a semilattice.
The converse also holds; we can induce a partial order on a semilattice.
%
\begin{theorem} [A semilattice and its induced partial order]
\label{basicTheorem}
Let $(L,\LTCjoin,\bot)$  be a semillatice. 
Then there exists a unique partial order on $L$
such that $a \LTCjoin b$ is the join of $\set{a,b}$, and
$\bot$ is the least element of $L$. 
\end{theorem}
%
\begin{proof}
Clearly, if such a partial order exists, 
we must have $a \leq_\LTCjoin b$ iff $a \LTCjoin b = b$. 
So we take this as a definition of $\leq_\LTCjoin$. 

\begin{enumerate}
\item {\em $\leq_\LTCjoin$ is a partial order on $L$}:
\begin{description}
\item [reflexiveness] 
we deduce $a \leq_\LTCjoin a$ from idempotency: $a \LTCjoin a = a$.
\item [antisymmetry]  $a = a \LTCjoin b = b$. 
\item [transitivity]  
Suppose $a \leq_\LTCjoin b$ and $b \leq_\LTCjoin c$;
i.e. $a \vee b = b$ and $b \vee c = c$. 
Then we have $a \leq_\LTCjoin c$ as follows: 
\[ a \vee c = a \vee (b \vee c) = (a \vee b) \vee c = b \vee c = c \mbox{.}\]
\end{description}
%
The above proof is pictorially represented as; 
\begin{prooftree}
	\AxiomC{$a \vee a = a$} 
    \UnaryInfC{$a \leq_\LTCjoin a$}
\DisplayProof \hskip 48pt
	\AxiomC{$a \leq_\LTCjoin b$} \UnaryInfC{$a \vee b = b$}
    \AxiomC{$b \leq_\LTCjoin a$} \UnaryInfC{$b \vee a = a$}
    \BinaryInfC{$a = b$}
\DisplayProof \hskip 48pt
	\AxiomC{$a \leq_\LTCjoin b$} \UnaryInfC{$a \vee b = b$}
    \AxiomC{$b \leq_\LTCjoin c$} \UnaryInfC{$b \vee c = c$}
    \BinaryInfC{$(a \vee b) \vee c = c$}
    \UnaryInfC{$a \vee (b \vee c) = c$}
    \UnaryInfC{$a \vee c = c$}
    \UnaryInfC{$a \leq_\LTCjoin c$}
\end{prooftree}

%
\item {\em $a \vee b$ is the join of $\set{a,b}$}:
\begin{enumerate}
\item $a \vee b$ is an upper bound for both $a$ and $b$
\begin{prooftree}
\AxiomC{$a \LTCjoin  b = a \LTCjoin  b$}
\UnaryInfC{$(a \LTCjoin a) \LTCjoin  b = a \LTCjoin  b$}
\UnaryInfC{$a \LTCjoin (a \LTCjoin  b) = a \LTCjoin  b$}
\UnaryInfC{$a \fCenter a \LTCjoin  b$}
	\DisplayProof \hskip 96pt
\AxiomC{$a \LTCjoin  b = a \LTCjoin  b$}
\UnaryInfC{$ a \LTCjoin  b = a \LTCjoin  (b \LTCjoin b)$}
\UnaryInfC{$a \LTCjoin  b = (a \LTCjoin b) \LTCjoin b$}
\UnaryInfC{$b \fCenter a \LTCjoin  b$}
\end{prooftree}
%
\item $a \LTCjoin b$ is the least upper bound for $\set{a,b}$
%
\begin{prooftree}
\AxiomC{$a \fCenter c$}
\UnaryInfC{$a \LTCjoin c = c$}
\AxiomC{$b \fCenter c$}
\UnaryInfC{$b \LTCjoin c = c$}
\BinaryInfC{$ a \LTCjoin (b \LTCjoin c) = c$}
\UnaryInfC{$ (a \LTCjoin b) \LTCjoin c = c$}
\UnaryInfC{$ a \LTCjoin b \fCenter c$}
\end{prooftree}
%
\item {\em $\bot$ is the least element}:
\begin{prooftree}
\AxiomC{$a \LTCjoin \LTCbot = a$}
\UnaryInfC{$ \LTCbot \fCenter a$}
\end{prooftree}
\end{enumerate}
\end{enumerate}
\end{proof}

The theorem says 
that the two definitions of a semilattice are equivalent. 
One may freely use both style in constructing proofs.

\begin{definition} [lattice]
A {\em lattice} is a quintuple $(L,\LTCjoin,\LTCbot,\LTCmeet,\LTCtop)$.
The following conditions hold.
\begin{enumerate}
\item {\em join-semilattice}: $(L,\LTCjoin,\LTCbot)$ is a join-semilattice,
where $a \leq_\LTCjoin b$ iff $b = a \LTCjoin b$. 
\item {\em meet-semilattice}: $(L,\LTCmeet,\LTCtop)$ is a meet-semilattice,
where $a \leq_\LTCmeet b$ iff $a \LTCmeet b = a$. 
\item {\em absorptive law}:  
for every $a,b \in L$ 
we have $a = a \LTCmeet (a \LTCjoin b) $
and   $(a \LTCmeet b) \LTCjoin b = b$.
\end{enumerate}
\end{definition}

\def\fCenter{\leq}

\begin{fact}
The two partial orders on $L$ ($\leq_\LTCjoin$,$\leq_\LTCmeet$) agree.
\end{fact}
\begin{proof}
\begin{prooftree}
\AxiomC{$a = a \LTCmeet (a \LTCjoin b) $}
\AxiomC{$a \leq_\LTCjoin b$}
\UnaryInfC{$b = a \LTCjoin b $}
\BinaryInfC{$a = a \LTCmeet b $}
\UnaryInfC{$a \leq_\LTCmeet b$}
	\DisplayProof \hskip 24pt
\AxiomC{$a \leq_\LTCmeet b$}
\UnaryInfC{$a \LTCmeet b = a $}
\AxiomC{$(a \LTCmeet b) \LTCjoin b = b $}
\BinaryInfC{$a \LTCjoin b = b $}
\UnaryInfC{$a \leq_\LTCjoin b$}
\end{prooftree}
\end{proof}
%%%
The absorptive laws distinguish a lattice from an arbitrary pair of semilattices. It assure that the partial orders on $L$ induced by the two semilattices are opposite to each other.
%%%

\begin{table}[t]
\LTCposetRules
\LTCjoinRules
\LTCmeetRules
\caption{axioms and inference rules for lattice}
\label{LTCsystem}
\noindent\makebox[\linewidth]{\rule{\textwidth}{0.4pt}}
\end{table}





\begin{table}[t]
\[
\begin{array}{ccc}

\begin{array}{c}
a \LTCjoin a = a \\
a \LTCjoin b = b \LTCjoin a \\
a \LTCjoin (b \LTCjoin c) = (a \LTCjoin b) \LTCjoin c \\
a \LTCjoin \LTCbot = a 
\end{array}
&
\begin{array}{c}
a \LTCmeet a = a \\
a \LTCmeet b = b \LTCmeet a \\
a \LTCmeet (b \LTCmeet c) = (a \LTCmeet b) \LTCmeet c \\
a \LTCmeet \LTCtop = a 
\end{array}
&
\begin{array}{c}
a = a \LTCmeet (a \LTCjoin b) \\
(a \LTCmeet b) \LTCjoin b = b
\end{array}
\end{array}
\]
\caption{axiomatic identities for lattices}
\label{LTCaxiomSystem}
\noindent\makebox[\linewidth]{\rule{\textwidth}{0.4pt}}
\end{table}

\def\fCenter{\leq}

\section{distibutive lattice}
%
% distributive lattice
%
Since lattices come with two binary operations, 
it is natural to ask whether one of them distributes over the other, 
i.e. whether one or the other of the following dual laws holds 
for every three elements $a, b, c$ of $L$:

\begin{enumerate}
\item {\em Distributivity of $\LTCjoin$ over $\LTCmeet$}:
$ c \LTCjoin (a \LTCmeet  b) = (c \LTCjoin a) \LTCmeet   (c \LTCjoin b)$
\item {\em Distributivity of $\LTCmeet$   over $\LTCjoin$}:
$c \LTCmeet  (a\LTCjoin b) = (c\LTCmeet  a) \LTCjoin  (c\LTCmeet  b)$
\end{enumerate}

\begin{definition}[distributive lattice]\label{distributiveLattice}
A lattice $(L,\LTCjoin,\LTCmeet,\LTCbot,\LTCtop)$ is a distributive lattice 
if it satisfies the avobe two {\em distributive laws}.
\end{definition}

The following definition and lemma shows that one is enough 
to define distributivity; the other (its dual) is automatically holds. 

\begin{lemma}[dual of distributive law holds]
If the distributive law holds in a lattice,  then so does its dual.
\end{lemma}
\begin{proof}
\begin{eqnarray*}
   (c \LTCjoin a) \LTCmeet (c \LTCjoin b)
= (c \LTCmeet c) \LTCjoin (c \LTCmeet b) \LTCjoin  (a \LTCmeet c) \LTCjoin (a \LTCmeet b) 
&=&  c \LTCjoin (a \LTCmeet b)  \\
   (c \LTCmeet a) \LTCjoin (c \LTCmeet b)
=  (c \LTCjoin c) \LTCmeet (c \LTCjoin b) \LTCmeet  (a \LTCjoin c) \LTCmeet (a \LTCjoin b) 
&=&  c \LTCmeet (a \LTCjoin b)
\end{eqnarray*}
\end{proof}

\subsection{Boolean Algebras} 

\begin{definition}[boolean algebra]
A {\em  Boolean algebra} is a sextuple 
$(L,\LTCjoin,\LTCbot,\LTCmeet,\LTCtop,\lnegP{\mbox{\textendash}})$.
The following conditions hold:
\begin{enumerate}
\item {\em distributive lattice}: 
$(L,\LTCjoin,\LTCbot,\LTCmeet,\LTCtop)$ is a distributive lattice,
\item  {\em complement law}: for each element $a$, 
there exists at most one  $\lneg{a}$  satisfying  
$a \LTCjoin \lneg{a} = \LTCtop$ and $a \LTCmeet \lneg{a} = \LTCbot$.  
\end{enumerate}
\end{definition}

\section{Heyting  Algebras}

\begin{definition}[heyting algebra]
A {\em  Heyting algebra} is a sextuple 
$(L,\LTCjoin,\LTCbot,\LTCmeet,\LTCtop,\RLimp)$.
The following conditions hold:
\begin{enumerate}
\item {\em lattice}: 
$(L,\LTCjoin,\LTCbot,\LTCmeet,\LTCtop)$ is a lattice,
\item  {\em residual law}: for each pair of elements $(a,b)$, 
there exists an element $a \LTCimp b$  such that  
$c \fCenter a \LTCimp b$ iff $a \LTCmeet c \fCenter b$.  
The inference rule can also be used as upside down.  Pictorially,
\hskip -7cm
\begin{prooftree}
	\AxiomC{$c \LTCmeet a \fCenter b$}
    \UnaryInfC{$c \fCenter a \RLimp b$}
 \DisplayProof \hskip 96pt   
    \AxiomC{$c \fCenter a \RLimp b$}
    \UnaryInfC{$c \LTCmeet a \fCenter b$}
\end{prooftree}
\end{enumerate}
\end{definition}





\begin{table}[t]
\[
\begin{array}{c}
c \LTCmeet (a \LTCjoin b) = (c \LTCmeet a) \LTCjoin (c \LTCmeet b) \\
a \LTCjoin \lneg{a} = \LTCtop \\
a \LTCmeet \lneg{a} = \LTCbot \\
\end{array}
\]
\caption{axiomatic identities for boolean algebras}
\label{BAaxiomSystem}
\noindent\makebox[\linewidth]{\rule{\textwidth}{0.4pt}}
\end{table}

\section{Commutative residuated lattices}

\begin{definition}[commutative residuated lattice]
A commutative {\em  residuated lattice} is a octaple 
$(L,\vee,\bot,\wedge,\top,\RLimp,\RLotimes,\RLone)$.
The following conditions hold:
\begin{enumerate}
\item {\em lattice}: 
$(L,\vee,\bot,\wedge,\top)$ is a lattice,
\item {\em commutative monoid}: 
$(L,\RLotimes,\RLone)$ is a commutative monoid,
\item  {\em residual law}: for each pair of elements $(a,b)$, 
there exists an element $a \RLimp b$  such that residual law holds:
\hskip -7cm
\begin{prooftree}
	\AxiomC{$c \RLotimes a \fCenter b$}
    \UnaryInfC{$c \fCenter a \RLimp b$}
 \DisplayProof \hskip 96pt   
    \AxiomC{$c \fCenter a \RLimp b$}
    \UnaryInfC{$c \RLotimes a \fCenter b$}
\end{prooftree}

\end{enumerate}
\end{definition}



\def\fCenter{\LLvdash}

\begin{table}[ht]
\begin{center}
\begin{prooftree}
\AxiomC{$A \fCenter A$}  
\RightLabel{(Weakening)} \UnaryInfC  {$A \fCenter B,A$}
\RightLabel{(R $\LLimp$)} \UnaryInfC  {$ \fCenter A \LLimp B,A$}
\AxiomC{A \fCenter A}
\RightLabel{(L $\LLimp$)}
\BinaryInfC{$ (A \LLimp B) \LLimp A \fCenter A,A$}
\RightLabel{(Contraction)}  \UnaryInfC{$ (A \LLimp B) \LLimp A \fCenter A$}
\RightLabel{(R $\LLimp$)}  \UnaryInfC{$  \fCenter ((A \LLimp B) \LLimp A) \LLimp A$}
\end{prooftree}
\end{center}
\caption{pierce's law}
\end{table}


\def\fCenter{\leq}

\begin{table}[ht]
\begin{center}
\begin{prooftree}
\AxiomC{$a \fCenter a$}  
\UnaryInfC  {$a \fCenter b \LTCjoin a$}
\UnaryInfC  {$ \LTCtop \fCenter (a \RLimp b) \LTCjoin a$}
\AxiomC{$(a \RLimp b) \RLimp a  \fCenter (a \RLimp b) \RLimp a$}
\UnaryInfC{$(a \RLimp b) \LTCmeet ((a \RLimp b) \RLimp a)  \fCenter   a$}
\BinaryInfC{$ (a \RLimp b) \RLimp a \fCenter a \LTCjoin a$}
\UnaryInfC{$ (a \RLimp b) \RLimp a \fCenter a$}
\UnaryInfC{$  \LTCtop \fCenter ((a \RLimp b) \RLimp a) \RLimp a$}
\end{prooftree}
\end{center}
\caption{proof of pierce's law in boolean algebra}
\end{table}

\subsection{A monad on a poset is precisely a closure operator}

\begin{fact}
From any poset $P$  we can define a category.
\end{fact}

\begin{proof}
We take as objects the elements of $P$, 
and when $x \leq y$, 
we have a unique arrow $x \rightarrow y$. 
\end{proof}

The arrow is usually left unnamed, since there's at most one for any pair of objects. 

\begin{enumerate}
    \item Reflexivity ensures the existence of $\mbox{id}_x$, 
as the unique arrow $x \leq x$.
    \item Transitivity ensures composites,
where  $(x \rightarrow y) ; (y \rightarrow z)$ 
is the unique arrow guaranteed by $x \leq z$. 
\end{enumerate}

\begin{fact}
Any diagram which we can draw in a poset commutes, since any parallel arrows are equal.
\end{fact}

\begin{proposition}
A monad on a poset is precisely a closure operator. 
\end{proposition}

\begin{proof}
\begin{enumerate}
    \item $T : C \rightarrow C$ is a functor: we must have $x \leq y$ implies $\bar{x} \leq \bar{y}$ . That is,  $T$ is monotone.
    \item There is a natural transformations 
$\eta : \mbox{1}_{\sf{C}} \Rightarrow T$.
Such a natural transformation is given by an arrow $x \rightarrow T(x)$  for all $x \in C$. 
That is, for all  $x \in C$ we have $x \leq T(x)$.
 That is, $T$ is extensive.
    \item 
There is a natural transformation and $\mu : T ; T \Rightarrow T$. As in the last point, this becomes for all $x \in C$  we have $T(T(x))$. 
That is, $T$ is idempotent.
\end{enumerate}
\end{proof}

\subsection{A T-algebra for a monad $T$  is a closed element}

\begin{definition}
An algebra for a monad $T$ on $C$ is an object $x \in C$ 
together with an arrow $T(x) \rightarrow x$  satisfying diagrams (that we again don't care about). 
\end{definition}

That is, in a poset, a T-algebra for a monad $T$ is an element 
 with $T(x) \leq x$  — a closed element.
 
\subsection{Prorders and Posets}

 \begin{definition} [Preorders]
A category $P$ is a {\em preorder} if, given objects $p$ and $q$, there is at most one arrow $p \rightarrow q$. 
\end{definition}

\begin{definition} [Skeletal category] 
A category is called {\em skeletal} when any two ismorphic objects are identical. 
\end{definition}

In any preorder $P$, 
define a binary relation $\leq$ on the objects of $P$, 
with $p \leq q$ if and only if there is an arrow $p \rightarrow q$ in $P$.

\begin{enumerate}
    \item  This binary relation is reflexive, 
    because there is an identy arrow $p \rightarrow p$. 
    \item It is also transitive, 
    because arrows $f: p \rightarrow q$ and $g: q \rightarrow r$ can be composed $ (f ; g) : p \rightarrow r$. 
\end{enumerate}

Conversely, any set $P$ with a reflexive, transitive binary relation $\leq$ determines a preorder,
in which the arrows $p \rightarrow q$ are exactly those ordered pairs 
$\langle p,q \rangle$ for which $p \leq q$. 
\begin{enumerate}
    \item Since the relation is  transitive, there is a unique way of composing these arrows.
    \item Since it is reflexive, there are the necessary indety arrows. 
\end{enumerate}

\begin{definition} [Posets]
A preorder  $P$ is a poset if it has the added axiom that $p \leq q$ and $q \leq p$ imply $p = q$.
\end{definition}

In other words, a skeletal preorder is a poset. 


 
%\bibliography{BiB/LinearLogicPfenning,BiB/AttackTree,BiB/algebra,BiB/PaulTaylor} 
\end{document}
